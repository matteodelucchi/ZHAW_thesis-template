%----------------------------------------------------------------------------------------
% SECTION 2
%----------------------------------------------------------------------------------------

\section{Test Data}
\label{test_data}

There are a few sources for test-data of XPS-spectra. A main database was found on XPSlibrary, provided by TXL. The data was kindly sent to us in a readable text-format from B. Vincent Crist. From this database, > 400 spectra were used for the test dataset. In addition, the XPSSurfA on CMSS Hub from the Australian University La Trobe contains more than 1700 spectra of which $\approx$ 50 survey spectra were used for the evaluation of the model. The test data from the XPSSurfA-database is stored in web form, accessible as a zip-file, a script for automated data-acquisition was written, which can be found in Appendix \ref{xpslibrary_webscraper}.
Overall, we obtained $\nexperimentalspectra$ experimental spectra, consisting of $\nelementalspectra$ elemental spectra, $\nmultispectra$  multi-component (compound) spectra, and $\noxidespectra$ oxide spectra.
As spectra with identified two-layer systems were not found on publicly available databases, we assumed that pure elemental compounds are a two-layer system of the same compounds.

For multi-component spectra, $\nmultispectra$ experimental spectra were obtained from the libraries mentioned above. The spectra were renamed manually to facilitate the automatized label-readability, according to the scheme ${E_{1}\textunderscore X_{1} 	\textunderscore E_{2} 	\textunderscore X_{2} 	\textunderscore Filename}$, where $E_{1}$ is the first elemental component and $X_{1}$ is an integer providing its relative concentration. If experimental spectra had ratios indicated (eg. 25:75), these values were multiplied with the compounds. Thus, a file with the components \ch{Na2O-SiO2} (25:75) will be encoded with filename Na \textunderscore 50 \textunderscore O \textunderscore 175 \textunderscore Si \textunderscore 75 \textunderscore Filename. The one-hot encoding into vectors is then done with a function which parses the filename, it can be found in Appendix \ref{code:base}.

\subsection{Experimental depth profiles using sputtering}
To test the models' capabilities to predict depth profiles of spectra, sputtering experiments were conducted using Copper and Palladium (Cu/Pd). Another dataset was provided by a group at Empa, consisting of Aluminum with oxidation layers. Because this is not represented in the training data, this is a valuable test set for inferring the generalisation capabilities of the model. Figure \ref{fig:exp_samples} shows the underlying structure of the test dataset.

\begin{figure}
    \centering
    \includegraphics{Figures/experimental_samples.png}
    \caption{Caption}
    \label{fig:enter-label}
\end{figure}


\subsection{Data preprocessing}

As is typical for data acquired from laboratories, XPS data comes in various formats and measurement parameters. Apart from XPS-specific parameters, the resolution and the range mostly influence the evaluation of spectra with the deep-learning framework. Thus, an automated workflow for the spectra pre-processing is introduced to provide a valid input shape to the model.
Although most wide-survey spectra are measured with a comparable energy-range, it should be identical to the training data for prediction. Thus, any signal outside the specified binding energy range (0-1000 eV) is cut off and the discrete measurement points inside the range are interpolated (if less than 1024) or subsampled (if more than 1024) by using a cubic-spline sampling method to match 1024 points. The most common format is the VAMAS format, established in 1988 by Dench et al \cite{dench_vamas_1988}. It has been slightly modified to the ISO Standard \cite{1400-1700_iso_nodate}(ISO 14976:1998) and thus is very popular and readable by most XPS applications. A python library found on the python package index PyPi called \emph{vamas} \cite{krinninger_vamas_nodate} was used under slight adaptations due to version updates of related packages. It allows reading and extracting data from Vamas files in a object-oriented manner.
Most experimental files obtained contain so-called blocks, which are sub-experiments on the same sample. In the different blocks, we usually find different analysis regions, such as a specific region for an atomic orbital, or the survey spectrum.

\begin{figure}
    \centering
    \includegraphics[width=\textwidth]{Figures/preprocessing_routine.png}
    \caption{Preprocessing routine result of experimental spectra (Aluminum). Experimental spectrum before (blue) and after (red) the preprocessing routine. Spectra are cut off at 1000 eV as shown in a) and at 0 eV. Slight changes to the peak shape are a result of the resampling method as shown in b).}
    \label{fig:preproc_routine}
\end{figure}

The preprocessing results of a selection of experimental spectra were then assessed, as shown in Figure \ref{fig:preproc_routine}. The Figure shows the cropping of spectra which go above and under the defined limit (a). Furthermore, slight changes in the peak shape and its position due to the resampling method used are shown (b). The python module used to preprocess the experimental data can be found in Appendix \ref{AppendixA}.


