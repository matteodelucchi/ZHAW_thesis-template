%----------------------------------------------------------------------------------------
% SECTION 2
%----------------------------------------------------------------------------------------

\section{Test Data}
\label{test_data}

There are a few sources for test-data of XPS-spectra. A main database was found on XPSlibrary, provided by TXL. The data was kindly sent to us in a readable text-format from B. Vincent Crist. From this database, > 400 spectra were used for the test dataset. In addition, the XPSSurfA on CMSS Hub from the Australian University La Trobe contains more than 1700 spectra of which $\approx$ 50 survey spectra were used for the evaluation of the model. The test data from the XPSSurfA-database is stored in web form, accessible as a zip-file, a script for automated data-acquisition was written, which can be found in Appendix \ref{xpslibrary_webscraper}.
Overall, we obtained $\nexperimentalspectra$ experimental spectra, consisting of $\nelementalspectra$ elemental spectra, $\nmultispectra$  multi-component (compound) spectra, and $\noxidespectra$ oxide spectra.
As spectra with identified two-layer systems were not found on publicly available databases, we assumed that pure elemental compounds are a two-layer system of the same compounds.

\subsection{Experimental depth profiles using sputtering}
To test the models' capabilities to predict depth profiles of spectra, sputtering experiments were conducted using Copper on Tungsten and Copper and Palladium.


\subsection{Data preprocessing}

As is typical for data acquired from laboratories, XPS data comes in various formats and measurement parameters. Apart from XPS-specific parameters, the resolution and the range mostly influence the evaluation of spectra with the deep-learning framework. Thus, an automated workflow for the spectra pre-processing is introduced to provide a valid input shape to the model.
Although most wide-survey spectra are measured with a comparable energy-range, it should be identical to the training data for prediction. Thus, any signal outside the specified binding energy range (0-1000 eV) is cut off and the discrete measurement points inside the range are interpolated (if less than 1024) or subsampled (if more than 1024) by using a uniformly distributed sampling method to match 1024 points. Lastly, because this resampling of the data can shift the spectrum, we adjust the maximum peak to be at the exact same binding energy position. Although this can, in turn, shift other peaks, we minimize the impact on our most important peak with the maximal intensity. 
The most common format is the VAMAS format, established in 1988 by Dench et al \cite{dench_vamas_1988}. It has been slightly modified to the ISO Standard \cite{1400-1700_iso_nodate}(ISO 14976:1998) and thus is very popular and readable by most XPS applications. A python library found on the python package index PyPi called vamas \cite{krinninger_vamas_nodate} was used under slight adaptations due to version updates of related packages. It allows reading and extracting data from Vamas files in a object-oriented manner.
Most experimental files obtained contain so-called blocks, which are sub-experiments on the same sample. In the different blocks, we usually find different analysis regions, such as a specific region for an atomic orbital, or the survey spectrum.

\begin{figure}
    \centering
    \includegraphics[width=\textwidth]{Figures/preprocessing_routine.png}
    \caption{Preprocessing routine result of experimental spectra (Aluminum). Experimental spectrum before (blue) and after (red) the preprocessing routine. Spectra are cut off at 1000 eV as shown in a) and at 0 eV. Peak shifts (-0.7 eV) as shown in b) are a result of the resampling method.}
    \label{fig:preproc_routine}
\end{figure}

The preprocessing results of a selection of experimental spectra were then assessed, as shown in Figure \ref{fig:ex_vs_sim}.

\begin{figure}
    \includegraphics[width=\textwidth]{Figures/Fe_XPS.png}
    \caption{Comparison of experimental \& preprocessed vs simulated spectra of elemental iron}
    \label{fig:ex_vs_sim}
    \centering
\end{figure}

The python module used to preprocess the experimental data can be found in \nameref{AppendixA}.