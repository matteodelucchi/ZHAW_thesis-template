%%%%%%%%%%%%%%%%%%%%%%%%%%%%%%%%%%%%%%%%%
% This is a template for LaTeX thesis (M.Sc. or PhD) at ZHAW.
% 
% ZHAW thesis template downloaded from:
% https://github.com/matteodelucchi/ZHAW_thesis-template
% 
% University specific changes were made by:
% Matteo Delucchi
%
% Based on a template downloaded from:
% http://www.LaTeXTemplates.com
% 
% Version 2.x major modifications by:
% Vel (vel@latextemplates.com)
%
% This template is based on a template by:
% Steve Gunn (http://users.ecs.soton.ac.uk/srg/softwaretools/document/templates/)
% Sunil Patel (http://www.sunilpatel.co.uk/thesis-template/)
%
% Template license:
% CC BY-NC-SA 3.0 (http://creativecommons.org/licenses/by-nc-sa/3.0/)
%
%%%%%%%%%%%%%%%%%%%%%%%%%%%%%%%%%%%%%%%%%

%----------------------------------------------------------------------------------------
%	PACKAGES AND OTHER DOCUMENT CONFIGURATIONS
%----------------------------------------------------------------------------------------

\documentclass[
11pt, % The default document font size, options: 10pt, 11pt, 12pt
%oneside, % Two side (alternating margins) for binding by default, uncomment to switch to one side
english, % ngerman for German
singlespacing, % Single line spacing, alternatives: onehalfspacing or doublespacing
%draft, % Uncomment to enable draft mode (no pictures, no links, overfull hboxes indicated)
%nolistspacing, % If the document is onehalfspacing or doublespacing, uncomment this to set spacing in lists to single
%liststotoc, % Uncomment to add the list of figures/tables/etc to the table of contents
%toctotoc, % Uncomment to add the main table of contents to the table of contents
%parskip, % Uncomment to add space between paragraphs
%nohyperref, % Uncomment to not load the hyperref package
headsepline, % Uncomment to get a line under the header
%chapterinoneline, % Uncomment to place the chapter title next to the number on one line
%consistentlayout, % Uncomment to change the layout of the declaration, abstract and acknowledgements pages to match the default layout
]{MastersDoctoralThesis} % The class file specifying the document structure

\usepackage[utf8]{inputenc} % Required for inputting international characters
\usepackage[T1]{fontenc} % Output font encoding for international characters
\usepackage{xcolor}

%\usepackage[x-5pg]{pdfx} %for PDF-X support. Has problems with xcolor and hyperref
%\usepackage{subfigure}
\usepackage{caption}
\usepackage{subcaption}

\usepackage{mathpazo} % Use the Palatino font by default

\usepackage[backend=biber,  % Use the bibtex backend with the authoryear citation style (which resembles APA)
sorting=none, % numbers the reference in order of their appearance in the document. Troubleshoot with deleting all *.aux and *.bbl files and rebuild.
style=numeric-comp, % Compact numeric citation scheme intended for in-text citations. Compact citations replace [1, 2, 3] by [1-3].
% style=authoryear-comp, % Compact author-year citation scheme. To be used in conjunction with the author-year bibliography style. Prints Doe 1992, 1995 instead of Doe 1992, Doe 1995.
natbib=true]{biblatex}

\addbibresource{example.bib} % The filename of the bibliography

\usepackage[autostyle=true]{csquotes} % Required to generate language-dependent quotes in the bibliography

\usepackage{pdfpages}

\usepackage{todonotes}
\setlength{\marginparwidth}{2.5cm} % uncomment this if the todonotes are out of margins (cut off page)

\usepackage{listings}


\definecolor{codegreen}{rgb}{0,0.6,0}
\definecolor{codegray}{rgb}{0.5,0.5,0.5}
\definecolor{codepurple}{rgb}{0.58,0,0.82}
\definecolor{backcolour}{rgb}{0.921, 0.929, 0.937} %0.95,0.95,0.92

\lstdefinestyle{mystyle}{
backgroundcolor=\color{backcolour},   
commentstyle=\color{codegreen},
keywordstyle=\color{magenta},
numberstyle=\tiny\color{codegray},
stringstyle=\color{codepurple},
basicstyle=\ttfamily\footnotesize,
breakatwhitespace=false,         
breaklines=true,                 
captionpos=b,                    
keepspaces=true,                 
numbers=left,                    
numbersep=5pt,                  
showspaces=false,                
showstringspaces=false,
showtabs=false,                  
tabsize=2
}

\lstset{style=mystyle}

\usepackage{makecell} % formatting tables





%----------------------------------------------------------------------------------------
%	MARGIN SETTINGS
%----------------------------------------------------------------------------------------

\geometry{
paper=a4paper, % Change to letterpaper for US letter
inner=2.5cm, % Inner margin
outer=3.8cm, % Outer margin
bindingoffset=.5cm, % Binding offset
top=1.5cm, % Top margin
bottom=1.5cm, % Bottom margin
%showframe, % Uncomment to show how the type block is set on the page
}


%----------------------------------------------------------------------------------------
%	THESIS INFORMATION
%----------------------------------------------------------------------------------------

\thesistitle{Thesis Title} % Your thesis title, this is used in the title and abstract, print it elsewhere with \ttitle
\supervisor{Dr. James \textsc{Smith}} % Your supervisor's name, this is used in the title page, print it elsewhere with \supname
\supervisorsec{Dr. Max \textsc{Miller}} % Your second supervisor's name, this is used in the title page, print it elsewhere with \supnamesec
\examiner{} % Your examiner's name, this is not currently used anywhere in the template, print it elsewhere with \examname
\degree{Doctor of Philosophy} % Your degree name (Doctor of Philosophy or Master of Science), this is used in the title page and abstract, print it elsewhere with \degreename
\author{John \textsc{Smith}} % Your name, this is used in the title page and abstract, print it elsewhere with \authorname
\addresses{} % Your address, this is not currently used anywhere in the template, print it elsewhere with \addressname

\subject{Biological Sciences} % Your subject area, this is not currently used anywhere in the template, print it elsewhere with \subjectname
\keywords{} % Keywords for your thesis, this is not currently used anywhere in the template, print it elsewhere with \keywordnames
\university{\href{https://www.zhaw.ch/en/university/}{Zurich University of Applied Sciences}} % Your university's name and URL, this is used in the title page and abstract, print it elsewhere with \univname
\universitygerman{\href{https://www.zhaw.ch/de/hochschule/}{Z{\"u}rcher Hochschule f{\"u}r Angewandte Wissenschaften}}% Your university's name in german and URL, this is used in the german abstract (Zusammenfassung), print it elsewhere with \univnameger
\department{\href{https://www.zhaw.ch/en/lsfm/institutes-centres/ias/}{Institute of Applied Simulation}} % Your department's name and URL, this is used in the title page and abstract, print it elsewhere with \deptname
\group{\href{https://www.zhaw.ch/en/lsfm/institutes-centres/ias/research-development/computational-genomics/}{Applied Computational Genomics}} % Your research group's name and URL, this is used in the title page, print it elsewhere with \groupname
\faculty{\href{https://www.zhaw.ch/de/lsfm/}{Life Sciences and Facility Management
}} % Your faculty's name and URL, this is used in the title page and abstract, print it elsewhere with \facname

\AtBeginDocument{
\hypersetup{pdftitle=\ttitle} % Set the PDF's title to your title
\hypersetup{pdfauthor=\authorname} % Set the PDF's author to your name
\hypersetup{pdfkeywords=\keywordnames} % Set the PDF's keywords to your keywords
}

\begin{document}

\frontmatter % Use roman page numbering style (i, ii, iii, iv...) for the pre-content pages

\pagestyle{plain} % Default to the plain heading style until the thesis style is called for the body content

%----------------------------------------------------------------------------------------
%	TITLE PAGE
%----------------------------------------------------------------------------------------

\begin{titlepage}
\begin{center}

\begin{figure}
\centering
\includegraphics[width=0.15\textwidth]{Figures/ZHAW_Logo.png} % Universtiy Logo, Adapted from: https://upload.wikimedia.org/wikipedia/commons/thumb/e/e6/ZHAW_Logo.svg/879px-ZHAW_Logo.svg.png
\end{figure}

\vspace*{.06\textheight}
{\scshape\LARGE \univname\par}\vspace{1.5cm} % University name
\textsc{\Large Doctoral Thesis}\\[0.5cm] % Thesis type

\HRule \\[0.4cm] % Horizontal line
{\huge \bfseries \ttitle\par}\vspace{0.4cm} % Thesis title
\HRule \\[1.5cm] % Horizontal line
 
\begin{minipage}[t]{0.4\textwidth}
\begin{flushleft} \large
\emph{Author:}\\
\href{http://www.johnsmith.com}{\authorname} % Author name - remove the \href bracket to remove the link
\end{flushleft}
\end{minipage}
\begin{minipage}[t]{0.4\textwidth}
\begin{flushright} \large
\emph{First Supervisor:} \\
\href{http://www.jamessmith.com}{\supname} % Supervisor name - remove the \href bracket to remove the link  
\end{flushright}
\begin{flushright} \large
\emph{Second Supervisor:} \\
\href{http://www.maxmiller.com}{\supnamesec} % Supervisor name - remove the \href bracket to remove the link  
\end{flushright}
\end{minipage}\\[3cm]
 
\vfill

\large \textit{A thesis submitted in fulfillment of the requirements\\ for the degree of \degreename}\\[0.3cm] % University requirement text
\textit{in the}\\[0.4cm]
\groupname\\\deptname\\[2cm] % Research group name and department name
 
\vfill

{\large \today}\\[4cm] % Date
%\includegraphics{Logo} % University/department logo - uncomment to place it
 
\vfill
\end{center}
\end{titlepage}

%----------------------------------------------------------------------------------------
%	DECLARATION PAGE
%----------------------------------------------------------------------------------------

\begin{declaration}
\addchaptertocentry{\authorshipname} % Add the declaration to the table of contents
COMMENT THIS SECTION IF THE \href{https://www.zhaw.ch/en/lsfm/study/studiweb/master-ls/masters-thesis/}{ORIGINAL COPY OF THE ZHAW DECLARATION OF ORIGINALITY} IS USED IN THE APPENDIX.\newline

\noindent I, \authorname, declare that this thesis titled, \enquote{\ttitle} and the work presented in it are my own. I confirm that:

\begin{itemize} 
\item This work was done wholly or mainly while in candidature for a research degree at this University.
\item Where any part of this thesis has previously been submitted for a degree or any other qualification at this University or any other institution, this has been clearly stated.
\item Where I have consulted the published work of others, this is always clearly attributed.
\item Where I have quoted from the work of others, the source is always given. With the exception of such quotations, this thesis is entirely my own work.
\item I have acknowledged all main sources of help.
\item Where the thesis is based on work done by myself jointly with others, I have made clear exactly what was done by others and what I have contributed myself.\\
\end{itemize}
 
\noindent Signed:\\
\rule[0.5em]{25em}{0.5pt} % This prints a line for the signature
 
\noindent Date:\\
\rule[0.5em]{25em}{0.5pt} % This prints a line to write the date
\end{declaration}

\cleardoublepage

%----------------------------------------------------------------------------------------
%	QUOTATION PAGE
%----------------------------------------------------------------------------------------

\vspace*{0.2\textheight}

% \noindent\enquote{\itshape I'm fascinated by the idea that genetics is digital. A gene is a long sequence of coded letters, like computer information. Modern biology is becoming very much a branch of information technology.}\bigbreak

% \hfill Richard Dawkins


% \vspace*{0.2\textheight}

% \noindent\enquote{\itshape A line is a dot that went for a walk.}\bigbreak

% \hfill Paul Klee

\noindent\enquote{\itshape You can’t connect the dots looking forward; you can only connect them looking backwards. So you have to trust that the dots will somehow connect in your future. You have to trust in something – your gut, destiny, life, karma, whatever. Because believing that the dots will connect down the road will give you the confidence to follow your heart even when it leads you off the well worn path; and that will make all the difference.
}\bigbreak
\hfill{Steve Jobs} \newline 
\strut\hfill{\tiny{(Stanford commencement speech, June 2005)}}

%----------------------------------------------------------------------------------------
%	ABSTRACT PAGE
%----------------------------------------------------------------------------------------

\begin{abstract}
\addchaptertocentry{\abstractname} % Add the abstract to the table of contents
The Thesis Abstract is written here (and usually kept to just this page). The page is kept centered vertically so can expand into the blank space above the title too\ldots

Structure it the in the following way: context, Need (what we have), Need (what we want), task, object of the document, findings, Conclusion and perspectives. 
\end{abstract}

%----------------------------------------------------------------------------------------
%	German ABSTRACT PAGE
%----------------------------------------------------------------------------------------

\begin{extraAbstract}
    \addchaptertocentry{\extraabstractname} % Add the abstract to the table of contents
Die Zusammenfassung der Dissertation wird hier geschrieben (und normalerweise nur auf dieser Seite gehalten). Die Seite wird vertikal zentriert, so dass sie sich auch in den leeren Raum über dem Titel ausdehnen kann\ldots

Strukturiere sie wie folgt: Kontext, Bedarf (was wir haben), Bedarf (was wir wollen), Aufgabe, Gegenstand des Dokuments, Ergebnisse, Schlussfolgerung und Perspektiven. 
    
\end{extraAbstract}
%----------------------------------------------------------------------------------------
%	ACKNOWLEDGEMENTS
%----------------------------------------------------------------------------------------

\begin{acknowledgements}
\addchaptertocentry{\acknowledgementname} % Add the acknowledgements to the table of contents
The acknowledgments and the people to thank go here, don't forget to include your project advisor\ldots
\end{acknowledgements}

%----------------------------------------------------------------------------------------
%	LIST OF CONTENTS/FIGURES/TABLES PAGES
%----------------------------------------------------------------------------------------

\tableofcontents % Prints the main table of contents

\listoffigures % Prints the list of figures

\listoftables % Prints the list of tables

%----------------------------------------------------------------------------------------
%	ABBREVIATIONS
%----------------------------------------------------------------------------------------

\begin{abbreviations}{ll} % Include a list of abbreviations (a table of two columns)

\textbf{LAH} & \textbf{L}ist \textbf{A}bbreviations \textbf{H}ere\\
\textbf{WSF} & \textbf{W}hat (it) \textbf{S}tands \textbf{F}or\\

\end{abbreviations}

%----------------------------------------------------------------------------------------
%	PHYSICAL CONSTANTS/OTHER DEFINITIONS
%----------------------------------------------------------------------------------------

\begin{constants}{lr@{${}={}$}l} % The list of physical constants is a three column table

% The \SI{}{} command is provided by the siunitx package, see its documentation for instructions on how to use it

Speed of Light & $c_{0}$ & \SI{2.99792458e8}{\meter\per\second} (exact)\\
%Constant Name & $Symbol$ & $Constant Value$ with units\\

\end{constants}

%----------------------------------------------------------------------------------------
%	SYMBOLS
%----------------------------------------------------------------------------------------

\begin{symbols}{lll} % Include a list of Symbols (a three column table)

$a$ & distance & \si{\meter} \\
$P$ & power & \si{\watt} (\si{\joule\per\second}) \\
%Symbol & Name & Unit \\

\addlinespace % Gap to separate the Roman symbols from the Greek

$\omega$ & angular frequency & \si{\radian} \\

\end{symbols}

%----------------------------------------------------------------------------------------
%	DEDICATION
%----------------------------------------------------------------------------------------

\dedicatory{For/Dedicated to/To my\ldots} 

%----------------------------------------------------------------------------------------
%	PREFACE
%----------------------------------------------------------------------------------------

\begin{preface}
    \addchaptertocentry{\prefacename} % Add the acknowledgements to the table of contents
    Explain the motivation for the topic in general. Go through each chapter and explain how they contribute to your research question.
    Check the thesis of Johannes John Carel Kuiper for a good example preface: \url{https://dspace.library.uu.nl/bitstream/handle/1874/90/full.pdf?sequence=2&isAllowed=y}
\end{preface}

%----------------------------------------------------------------------------------------
%	THESIS CONTENT - CHAPTERS
%----------------------------------------------------------------------------------------

\mainmatter % Begin numeric (1,2,3...) page numbering

\pagestyle{thesis} % Return the page headers back to the "thesis" style

% Include the chapters of the thesis as separate files from the Chapters folder
% Uncomment the lines as you write the chapters

\include{Chapters/Chapter1}
%% Indicate the main file. Must go at the beginning of the file.
% !TEX root = ../main.tex

%----------------------------------------------------------------------------------------
% CHAPTER 2
%----------------------------------------------------------------------------------------

\chapter{Code Listings}

\label{Chapter2} % For referencing the chapter elsewhere, use \ref{Chapter2} 

%----------------------------------------------------------------------------------------

The package \href{https://www.overleaf.com/learn/latex/Code\_listing}{\code{listings}} permits to easily include existing code. Simply use the command \verb|\lstinputlisting[language=name]{path/to/file}|. See \href{https://www.overleaf.com/learn/latex/Code\_listing#Supported\_languages}{here} for a list of supported programming languages.


\lstinputlisting[language=Python,caption=External file: code/example.py]{Code/example.py}

It is also possible to enter code directly into \LaTeX:

\begin{lstlisting}[language=C++]
#include <stdio>
void hello_world(void){
   std::cout << "Hello World!" << std::endl;
}    
\end{lstlisting}

Alternatively, one can use the syntax highlighting toolbox \href{https://pygments.org/}{\code{Pygments}} in combination with the \LaTeX-package \href{www.overleaf.com/learn/latex/Code\_Highlighting\_with\_minted}{\code{minted}}. It provides slightly better results, as the code will actually be parsed.

To install Pygments, use the following command. For \code{minted} to work properly, run the pdflatex tool with the flag \code{--shell-escape}. If you are using a TEX editor, you can modify the typesetting   command somewhere in the settings.

\begin{lstlisting}[language=bash]
# Make sure that Pygments is installed.
python -m pip install pygments

# Then add the --shell-escape flag to the command 
# that is used to compile your LaTeX code.
pdflatex --output-dir="$BUILD_DIR" \
         --file-line-error \
         --shell-escape \
         --synctex=1 "$1"
\end{lstlisting}



% Set the following line to \iftrue if minted is available on your system.
% See the above instructions to see how.

\iffalse
See below how the result looks like if minted is available on your system.

\begin{listing}[!ht]
\inputminted[linenos, bgcolor=codebackground, style=friendly]{python}{Code/example.py}
\caption{Example from external file, parsed using \code{Pygments}}
\end{listing}

\fi 
%\include{Chapters/Chapter3}
%\include{Chapters/Chapter4} 
%\include{Chapters/Chapter5} 

%----------------------------------------------------------------------------------------
%	THESIS CONTENT - APPENDICES
%----------------------------------------------------------------------------------------

\appendix % Cue to tell LaTeX that the following "chapters" are Appendices

% Include the appendices of the thesis as separate files from the Appendices folder
% Uncomment the lines as you write the Appendices

% !TEX root = ../main.tex

%----------------------------------------------------------------------------------------
% APPENDIX A
%----------------------------------------------------------------------------------------

\chapter{Code} % Main appendix title

\label{AppendixA} % For referencing this appendix elsewhere, use \ref{AppendixA}

\section{Modules}

\subsection{Base module}
\label{code:base}
\lstinputlisting[language=Python]{Code/base.py}

\subsection{Functions module}
\label{code:functions}
\lstinputlisting[language=Python]{Code/functions_tf.py}

\subsection{Preprocessing module}
\label{preprocessing}
\lstinputlisting[language=Python]{Code/preprocess.py}

\subsection{Prediction module}
\label{code:base}
\lstinputlisting[language=Python]{Code/modules/predict.py}

\subsection{Sessa Simulation module}
\label{Sessa_Module}
\lstinputlisting[language=Python]{Code/sessa_py.py}



\section{Notebooks}

\subsection{NIST Chemical Shift Database Webscraper}
\label{NIST_WebScraper}
\lstinputlisting[language=Python]{Code/req_async.py}

\subsection{XPSLibrary Webscraper}
\label{xpslibrary_webscraper}
\subsubsection{Grab folders from XPSSurfA}\label{grab-folders-from-xpssurfa}

\begin{lstlisting}[language=Python]
from bs4 import BeautifulSoup
import os
import requests
import urllib.request
import os
\end{lstlisting}

\begin{lstlisting}[language=Python]
number = 1
URL = f'https://cmsshub.latrobe.edu.au/xpsdatabase/xpsrecords/download_data_files/{number}'
req = requests.get(URL)
soup = BeautifulSoup(open('../../data/test_data/XPSSurfA/view-source_https___cmsshub.latrobe.edu.au_xpsdatabase_xpsrecords.html'),
                            'html.parser')
numbers = [int(p.get('href').split('/')[-1]) 
           for p in soup.findAll(class_="html-attribute-value html-external-link") 
           if 'view' in p.get('href')]
\end{lstlisting}

\begin{lstlisting}[language=Python]
for number in numbers:
    print(number)
    if os.path.isfile(f'../../data/test_data/XPSSurfA/{number}.zip'):
        print('already downloaded')
        continue
    URL = f'https://cmsshub.latrobe.edu.au/xpsdatabase/xpsrecords/download_data_files/{number}'
    file_name  = f'../../data/test_data/XPSSurfA/{number}.zip'
    # Download the file from `url` and save it locally under `file_name`:
    with urllib.request.urlopen(URL) as response, open(file_name, 'wb') as out_file:
        data = response.read() # a `bytes` object
        out_file.write(data)
\end{lstlisting}

\subsubsection{Unzip all downloaded files from
XPSSurfA}\label{unzip-all-downloaded-files-from-xpslibrary.com}

\begin{lstlisting}[language=Python]
import os
import zipfile

root_folder = '../../data/test_data/XPSSurfA'

# add //? before path if the path is too long otherwise it will throw an error

def extract_zip_files(root_folder):
    for foldername, subfolders, filenames in os.walk(root_folder):
        for filename in filenames:
            if filename.endswith('.zip'):
                zip_file_path = os.path.join(foldername, filename)
                print(os.path.join(foldername, os.path.splitext(filename)[0]))
                with zipfile.ZipFile(zip_file_path, 'r') as zip_ref:
                    zip_ref.extractall(os.path.join(foldername, os.path.splitext(filename)[0]))
\end{lstlisting}

\begin{lstlisting}[language=Python]
extract_zip_files(root_folder)
\end{lstlisting}

\subsubsection{Check files from XPSSurfA}\label{check-files-from-xpssurfa}

\begin{lstlisting}[language=Python]
files = []
for foldername, subfolders, filenames in os.walk(root_folder):
    # check if there is a *.vms file in the folder
    # print(foldername)
    if any([filename.endswith('.vms') for filename in filenames]):
        # print('yes')
        files.append(([foldername + '\\' +filename for filename in filenames if filename.endswith('.vms')][0]))
\end{lstlisting}

\begin{lstlisting}[language=Python]
len(files) # we have 121 test files
\end{lstlisting}

\begin{lstlisting}
121
\end{lstlisting}

\begin{lstlisting}[language=Python]
files = [f for f in files if len(f.split('_')) > 2 and len(f.split('_')[-1]) < 15 and not 'Cali' in f]
\end{lstlisting}

\begin{lstlisting}[language=Python]
files
\end{lstlisting}

\begin{lstlisting}
['../../data/test_data/XPSSurfA\\10\\Fe_Fe.vms',
 '../../data/test_data/XPSSurfA\\13\\Ar_Ar.vms',
 '../../data/test_data/XPSSurfA\\149\\Cu_Cu_Ultra.vms',
 '../../data/test_data/XPSSurfA\\15\\Mg_Mg.vms',
 '../../data/test_data/XPSSurfA\\150\\Ag_Ag_Ultra.vms',
 '../../data/test_data/XPSSurfA\\151\\Au_Au_Ultra.vms',
 '../../data/test_data/XPSSurfA\\154\\Nb_Nb.vms',
 '../../data/test_data/XPSSurfA\\16\\Ni_Ni.vms',
 '../../data/test_data/XPSSurfA\\17\\Mo_Mo.vms',
 '../../data/test_data/XPSSurfA\\18\\Ta_Ta.vms',
 '../../data/test_data/XPSSurfA\\2\\Au_Au.vms',
 '../../data/test_data/XPSSurfA\\20\\Al_Al.vms',
 '../../data/test_data/XPSSurfA\\22\\Si_Si.vms',
 '../../data/test_data/XPSSurfA\\3\\Ag_Ag.vms',
 '../../data/test_data/XPSSurfA\\4\\Pt_Pt.vms',
 '../../data/test_data/XPSSurfA\\5\\Cu_Cu.vms',
 '../../data/test_data/XPSSurfA\\7\\W_W.vms',
 '../../data/test_data/XPSSurfA\\9\\In_In.vms']
\end{lstlisting}


\subsection{Sessa Simulation Notebook}
\label{xpslibrary_webscraper}
\hypertarget{define-directories}{%
\section{Define directories}\label{define-directories}}

\begin{lstlisting}[language=Python]
from sessa_py import Experiment, Layer
import os
import numpy as np
import matplotlib.pyplot as plt
import itertools
import pandas as pd
import random
import subprocess
import sys
sys.path.append('../../../modules/')
import base
from tqdm.notebook import tqdm

root_dir = r'C:\Users\kochk\Documents\Git_Repos\Github\deep_xps'
sessa_dir = r"C:\Program Files (x86)\Sessa v2.2.0\bin\\"
\end{lstlisting}

\begin{lstlisting}[language=Python]
elements_sym = base.load_elem()
\end{lstlisting}

\hypertarget{build-experiments}{%
\section{Build experiments}\label{build-experiments}}

\hypertarget{experiments-for-permutations-of-elements-and-thicknesses}{%
\subsection{Experiments for permutations of elements and
thicknesses}\label{experiments-for-permutations-of-elements-and-thicknesses}}

\begin{lstlisting}[language=Python]
thicknesses = [10,20,30,40,50]
\end{lstlisting}

\begin{lstlisting}[language=Python]
perms = list(itertools.permutations(elements_sym, 2))
\end{lstlisting}

\hypertarget{separate}{%
\subsection{Separate}\label{separate}}

\begin{lstlisting}[language=Python]
# check if all files done and store not simulated files in a list
os.chdir(r'C:\Users\kochk\Documents\Git_Repos\Github\deep_xps\tasks\0\simulation')
perms = list(itertools.permutations(elements_sym, 2))

data_dir = '../../../data/simulation_data/depth_sep/'
files = os.listdir(data_dir)
not_sim = []
sim = []

for perm in perms:
    for thickness in thicknesses:
        if os.path.isfile(data_dir+f'{perm[0]}_{perm[1]}_{thickness}_separate_spectra.spcreg1.spc'):
            sim.append([perm, thickness])
        else:
            not_sim.append(perm)
\end{lstlisting}

\begin{lstlisting}[language=Python]
len(not_sim)
\end{lstlisting}

\begin{lstlisting}[language=Python]
exp_dir = 'sep_NO'
data_dir = rf'C:\Users\kochk\Documents\Git_Repos\Github\deep_xps\data\simulation_data\{exp_dir}'

for entry in tqdm(perms):
    for thickness in thicknesses:
        # go to next if file already exists
        if os.path.isfile(data_dir+f'\{entry[0]}_{entry[1]}_{thickness}_separate_spectra.spcreg1.spc'):
            continue
        else:
            f = Experiment([Layer(entry[0], 50), Layer(entry[1], thickness)],
                        name=f'{entry[0]}_{entry[1]}_{thickness}',
                        exp_dir=exp_dir,
                        root_dir=root_dir,
                        sessa_dir=sessa_dir,
                        contamination=None)
            f.simulate()
\end{lstlisting}

\hypertarget{experiments-for-oxidates-with-and-without-carbon-traces}{%
\subsection{Experiments for oxidates with and without carbon
traces}\label{experiments-for-oxidates-with-and-without-carbon-traces}}

\begin{lstlisting}[language=Python]
oxides = [
        '/Be/O/',
        '/Mg/O/',
        '/B2/O3/',
        '/Al2/O3/',
        '/Si/O2/',
        '/Sc2/O3/',
        '/Ti/O2/',
        '/Cr2/O3/',
        '/Mn/O2/',
        '/Fe2/O3/',
        '/Co/O/',
        '/Ni/O/',
        '/Cu/O/',
        '/Zn/O/',
        '/Ga2/O3/',
        '/Ge/O2/',
        '/As2/O3/',
        '/Y2/O3/',
        '/Zr/O2/',
        '/Nb2/O5/',
        '/Mo/O3/',
        '/Ru/O2/',
        '/Rh2/O3/',
        '/Pd/O/',
        '/Ag/O/',
        '/Cd/O/',
        '/In2/O3/',
        '/Sn/O2/',
        '/Sb2/O3/',
        '/Te/O2/',
        '/Hf/O2/',
        '/Ta2/O5/',
        '/W/O3/',
        '/Re/O3/',
        '/Ir/O2/',
        '/Pt/O/',
        '/Au2/O3/',
        '/Hg/O/',
        '/Tl2/O3/',
        '/Pb/O/',
        '/Bi2/O3/',
]
\end{lstlisting}

\begin{lstlisting}[language=Python]
for entry in tqdm(oxides):
    for thickness in thicknesses:
        f = Experiment(
                       layers=[Layer(entry.split('/')[1], 50), Layer(entry, thickness)],
                       root_dir= root_dir,
                       sessa_dir= sessa_dir,
                       exp_dir= 'oxides_NO',
                       contamination=None,
                       shifts_probability=0.8,
                       overwrite=False
                       )
        f.simulate()
\end{lstlisting}

\begin{lstlisting}[language=Python]
for entry in tqdm(oxides):
    for thickness in thicknesses:
        f = Experiment(
                       layers=[Layer(entry.split('/')[1], 50), Layer(entry, thickness)],
                       root_dir= root_dir,
                       sessa_dir= sessa_dir,
                       exp_dir= 'oxides_CO',
                       contamination=True,
                       shifts_probability=0.8,
                       overwrite=False
                       )
        f.simulate()
\end{lstlisting}

\hypertarget{build-multi-layer-system-for-depth-profiling}{%
\subsection{Build multi-layer system for
depth-profiling}\label{build-multi-layer-system-for-depth-profiling}}

\begin{lstlisting}[language=Python]
layer_thickness = 5 # Angstrom
grad_steps = layer_thickness
N_layers = 100 / grad_steps # 100 % divided by the gradient steps

def create_gradients(grad_steps, rev=False, N_layers=N_layers):
    gradient = np.arange(100+grad_steps, step=grad_steps)
    
    if rev is True: 
        rev_grad_matrix = np.repeat([np.flip(gradient)], axis=0, repeats=N_layers)
        for number, line in enumerate(rev_grad_matrix):
            if number == 0:
                continue
            rev_grad_matrix[number][-number:] = 0
        return rev_grad_matrix
    else:
        grad_matrix = np.repeat([gradient], axis=0, repeats=N_layers)
        for number, line in enumerate(grad_matrix):
            if number == 0:
                continue
            grad_matrix[number][-number:] = 0
        return grad_matrix
\end{lstlisting}

\begin{lstlisting}[language=Python]
rev = create_gradients(5, rev=True)
grad = create_gradients(5)
grad = np.arange(100+grad_steps, step=grad_steps) # 0 to 100 in steps of grad_steps
rev = np.flip(grad) # reverse the gradient
\end{lstlisting}

\begin{lstlisting}[language=Python]
iterator = tqdm(perms)
for elems in iterator:
    a = Experiment(
                name=f'{elems[0]}_{elems[1]}',
                layers=[(Layer(f'(/{elems[0]}/){grad[p]}(/{elems[1]}/){rev[p]}', thickness=5))
                        for p in range(len(grad))],
                root_dir=root_dir,
                sessa_dir=sessa_dir,
                exp_dir='grad_NO',
                etching=8,
                contamination=None)
    a.simulate()
\end{lstlisting}

\begin{lstlisting}[language=Python]
# elems[0] is the bulk, elem[1] is the top layer with variable thickness
for elems in tqdm(perms):
        for thickness in thicknesses:
                f = Experiment(name=f'{elems[0]}_{elems[0]}_{elems[1]}',
                       layers= [Layer(elems[0], 50), Layer(elems[0], elems[1])], 
                       root_dir=root_dir, 
                       sessa_dir=sessa_dir,
                       exp_dir='depth_CO_sep_new',
                       etching=None)
                f.simulate()
\end{lstlisting}

\hypertarget{check-if-all-files-done}{%
\subsubsection{Check if all files done}\label{check-if-all-files-done}}

\begin{lstlisting}[language=Python]
for elems in perms:
    for thickness in thicknesses:
        path = not(os.path.isfile('..\data\simfiles\depth_CO_sep\{elems[0]}_{elems[1]}_{thickness}.txt'))
    if path:
        print(f'..\data\simfiles\depth_CO_sep\{elems[0]}_{elems[1]}_{thickness}.txt')
        break
\end{lstlisting}

\begin{lstlisting}[language=Python]
[(i, os.path.isfile(f'C:/Users/kochk/Documents/Git_Repos/Github/deep_xps/data/simfiles/depth_CO_sep_new/{elems[0]}_{elems[1]}.txt')) 
 for i, elems in enumerate(perms) 
 if not os.path.isfile(f'C:/Users/kochk/Documents/Git_Repos/Github/deep_xps/data/simfiles/depth_CO_sep_new/{elems[0]}_{elems[1]}.txt')]
\end{lstlisting}

\begin{lstlisting}[language=Python]
for i, elems in enumerate(perms):
    for thickness in thicknesses:
        if not os.path.isfile(f'C:/Users/kochk/Documents/Git_Repos/Github/deep_xps/data/simfiles/depth_CO_sep_new/{elems[0]}_{elems[1]}_{thickness}.txt'):
            print(i, elems, thickness)
\end{lstlisting}

\begin{lstlisting}[language=Python]
# remove files
path='../data/simulation_data/depth_CO_sep/'
for file in os.listdir(path):
    if len(file.split('_')) == 6:
        os.remove(os.path.join(path,file))
        print(f'{file} removed.')
\end{lstlisting}

\begin{lstlisting}[language=Python]
os.chdir(r'C:\Users\kochk\Documents\Git_Repos\Github\deep_xps\tasks\0\simulation')
\end{lstlisting}

\begin{lstlisting}[language=Python]
# check if all files done and store not simulated files in a list

data_dir = '../../../data/simulation_data/grad_NO/'
files = os.listdir(data_dir)
not_sim = []
sim = []
etching_rates = [100, 90, 80, 70, 60]
for perm in perms:
    for etching in etching_rates:
        if os.path.isfile(data_dir+f'{perm[0]}_{perm[1]}_{etching}_etching_spectra.spcreg1.spc'):
            sim.append([perm, etching])
        else:
            not_sim.append(perm)
not_sim = [f for f in not_sim if not "Xe" in f]
\end{lstlisting}

\begin{lstlisting}[language=Python]
not_sim = pd.Series(not_sim).unique()
\end{lstlisting}

\hypertarget{one-element-layers}{%
\subsection{One-element layers}\label{one-element-layers}}

\begin{lstlisting}[language=Python]
for element in tqdm(elements_sym):
    if os.path.isfile(f'{root_dir}\\data\\simulation_data\\one_layer\\{element}_{element}_spectra.spc'):
        continue
    cmd_str = ['\\PROJECT LOAD SESSION "C:\Program Files (x86)\Sessa v2.2.0\\bin/Sessa_ini.ses"',
                    '\\SPECTROMETER SET RANGE 486.6:1486.6 REGION 1',
                    f'\\SAMPLE SET MATERIAL {element}',
                    # f'\\SAMPLE ADD LAYER /C/O/ THICKNESS {int(random.triangular(12, 24, 15))} ABOVE 0',
                    '\\MODEL SET CONVERGENCE 1.000e-02',
                    '\\MODEL SET SE true',
                    '\\MODEL SIMULATE',
f'\\MODEL SAVE SPECTRA "{root_dir}\\data\\simulation_data\\one_layer\\{element}_{element}_spectra.spc"']
    with open(f'{root_dir}\\data\\simfiles\\one_layer\\{element}_{element}.txt', 'w') as f:
        f.writelines('\n'.join(cmd_str))
    filename_abs = f'{root_dir}\\data\\simfiles\\one_layer\\{element}_{element}.txt'
    os.chdir(sessa_dir)
    SWHIDE = 0
    info = subprocess.STARTUPINFO()
    info.dwFlags = subprocess.STARTF_USESHOWWINDOW
    info.wShowWindow = SWHIDE
    p = subprocess.Popen('sessa.exe -s "%s"' % filename_abs, startupinfo=info)
    p.wait()
\end{lstlisting}

\hypertarget{mixture-compounds}{%
\section{Mixture compounds}\label{mixture-compounds}}

\begin{lstlisting}[language=Python]
from base import get_combinations

folder = 'multi_one_layer'
data_dir = rf'C:\Users\kochk\Documents\Git_Repos\Github\deep_xps\data\simulation_data\{folder}'
perms = get_combinations(elements_sym, number_of_layers=1, number_of_combinations=30_000, lower=2, upper=4) # get one layer
thickness = [random.choice([10,20,30,40,50]), 50] # two-layer systems always have second layer of 50 Angstrom
for entry in tqdm(list(perms)[0]):
    print(f'\t Going for entry {entry} with thickness {thickness}')
    exp = Experiment(layers=[Layer(entry[i] , thickness[i]) for i in range(len(entry))],
            exp_dir=folder,
            root_dir=root_dir,
            sessa_dir=sessa_dir,
            contamination=False,
            shifts_probability=0.6)
    exp.simulate()
\end{lstlisting}

\hypertarget{visualize}{%
\section{visualize}\label{visualize}}

\begin{lstlisting}[language=Python]
import matplotlib as mpl
import seaborn as sns
import pandas as pd

os.chdir(root_dir)

df = pd.read_csv('./data/Si_1_Zn_1/results/spectra.spcreg1.spc', sep='\s+', names=['Energy [eV]', 'Intensity'], skiprows=1)
df2 = pd.read_csv('./data/Zn_1_Si_1/results/spectra.spcreg1.spc', sep='\s+', names=['Energy [eV]', 'Intensity'], skiprows=1)
df.Intensity.plot()
df2.Intensity.plot()
\end{lstlisting}


\subsection{Training dataset creation \& preprocessing}
\label{train_data_generation}
\hypertarget{import-and-prepare}{%
\section*{Import and prepare}\label{import-and-prepare}}
\begin{lstlisting}[language=Python]
import sys
import glob
import pandas as pd
import numpy as np
import json
import pickle
from sklearn.model_selection import train_test_split
from sklearn.preprocessing import MultiLabelBinarizer

sys.path.append('../../modules') # add own modules
import preprocess, predict, functions_tf, base
\end{lstlisting}

\begin{lstlisting}[language=Python]
elements_db = json.loads(open('../../elements_sim.json').read())["elements"]
elements = [elem["symbol"] for elem in elements_db]
n_elements = len(elements)

mlb = MultiLabelBinarizer()
mlb.fit([elements])
test_size_ratio = 0.2
\end{lstlisting}

\hypertarget{training-data-for-top-bot-layer}{%
\section*{Training Data for top bot
layer}\label{training-data-for-top-bot-layer}}

\hypertarget{mixcont}{%
\subsection*{mixcont}\label{mixcont}}

\begin{lstlisting}[language=Python]
folders = [
            'grad_CO',              # depth-profiles with CO-adv.       with gradient layers
            'grad_NO',              # depth-profiles without CO-adv.    with gradient layers
            'sep_CO',               # depth-profiles with CO-adv.       with separated layers
            'sep_NO',               # depth-profiles without CO-adv.    with separated layers
            'one_layer_CO',         # one-layer simulations with CO-adv.
            'one_layer',            # one-layer simulations
            ]

files = [file for file in glob.glob(f'../../data/simulation_data/{folders[0]}/*.spc')]
for folder in folders[1:]:
    files.extend([file for file in glob.glob(f'../../data/simulation_data/{folder}/*.spc')])
\end{lstlisting}

\begin{lstlisting}[language=Python]
df = pd.concat([pd.read_csv(file,
                            sep='\s+', header=None, skiprows=1,
                            usecols=[1],
                            names=['_'.join(file.split('\\')[1].split('_')[:-1])]).T 
                                    for file in files]).T
df.to_pickle('../../data/df_mixcont.pkl') # save the df without preprocessing
\end{lstlisting}

\hypertarget{preprocess}{%
\subsubsection*{preprocess}\label{preprocess}}

\begin{lstlisting}[language=Python]
df = pd.read_pickle('../../data/df_mixcont.pkl') 
\end{lstlisting}

\begin{lstlisting}[language=Python]
df_norm = preprocess.MaxScale_df(df).reset_index(drop=True)  # each spectrum is scaled to 1
# reduce size to 1024 and add relative noise
df_noise = df_norm[::2].apply(lambda x: x+x*np.random.normal(0, np.random.randint(1,3)*0.01 , len(x)))
df_scaled = df_noise.T
df_scaled = df_scaled.dropna()
df_scaled = df_scaled.T.reset_index(drop=True)
\end{lstlisting}

\begin{lstlisting}[language=Python]
df_scaled.to_pickle('../../data/df_mixcont_scaled.pkl')  # save the normalized, scaled df
\end{lstlisting}

\hypertarget{top-layer-data}{%
\subsubsection*{Top Layer data}\label{top-layer-data}}

\begin{lstlisting}[language=Python]
df_scaled = pd.read_pickle('../../data/df_scaled.pkl')
\end{lstlisting}

\begin{lstlisting}[language=Python]
x_train, x_test, y_train, y_test = train_test_split(df_scaled.T.values,
                                                    df_scaled.columns.map(lambda x: x.split('_')[0]), # first part of the filename is the top label
                                                    test_size=0.3,
                                                    random_state=42)
\end{lstlisting}

\begin{lstlisting}[language=Python]
y_train = np.array([    
                        [
                            mlb.transform([[y_train[i]]])[0]
                        ] 
                            for i in range(len(y_train))
                        ])
y_test = np.array([ 
                       [
                            mlb.transform([[y_test[i]]])[0],
                       ] 
                            for i in range(len(y_test))
                        ])
\end{lstlisting}

\begin{lstlisting}[language=Python]
data = {
        'name': 'two-layer and one-layer systems, top labels',
        'x_train': x_train,
        'x_test': x_test,
        'y_train': y_train,
        'y_test': y_test
}
\end{lstlisting}

\begin{lstlisting}[language=Python]
pickle.dump(data, open('../../data/training_data/1/dataset_mixcont_top_layer.pkl', 'wb'))
\end{lstlisting}

\hypertarget{bot-layer-data}{%
\subsubsection*{Bot Layer data}\label{bot-layer-data}}

\begin{lstlisting}[language=Python]
df_scaled = pd.read_pickle('../../data/df_scaled.pkl')
\end{lstlisting}

\begin{lstlisting}[language=Python]
x_train, x_test, y_train, y_test = train_test_split(df_scaled.T.values,
                                                    df_scaled.columns.map(lambda x: x.split('_')[1]), # second part of the filename is the bot label
                                                    test_size=0.3,
                                                    random_state=42)
\end{lstlisting}

\begin{lstlisting}[language=Python]
y_train = np.array([    
                        [
                            mlb.transform([[y_train[i]]])[0]
                        ] 
                            for i in range(len(y_train))
                        ])
y_test = np.array([ 
                       [
                            mlb.transform([[y_test[i]]])[0],
                       ] 
                            for i in range(len(y_test))
                        ])
\end{lstlisting}

\begin{lstlisting}[language=Python]
data = {
        'name': 'two-layer and one-layer systems, bot labels',
        'x_train': x_train,
        'x_test': x_test,
        'y_train': y_train,
        'y_test': y_test
}
\end{lstlisting}

\begin{lstlisting}[language=Python]
pickle.dump(data, open('../../data/training_data/1/dataset_mixcont_bot_layer.pkl', 'wb'))
\end{lstlisting}

\hypertarget{cont}{%
\subsection*{cont}\label{cont}}

\begin{lstlisting}[language=Python]
folders = [
            'grad_CO',              # depth-profiles with CO-adv.       with gradient layers
            'sep_CO',               # depth-profiles with CO-adv.       with separated layers
            'one_layer_CO',         # one-layer simulations with CO-adv.
            ]

files = [file for file in glob.glob(f'../../data/simulation_data/{folders[0]}/*.spc')]
for folder in folders[1:]:
    files.extend([file for file in glob.glob(f'../../data/simulation_data/{folder}/*.spc')])
\end{lstlisting}

\begin{lstlisting}[language=Python]
df = pd.concat([pd.read_csv(file,
                            sep='\s+', header=None, skiprows=1,
                            usecols=[1],
                            names=['_'.join(file.split('\\')[1].split('_')[:-1])]).T 
                                    for file in files]).T
df.to_pickle('../../data/df_cont.pkl') # save the df without preprocessing
\end{lstlisting}

\hypertarget{preprocess-1}{%
\subsubsection*{preprocess}\label{preprocess-1}}

\begin{lstlisting}[language=Python]
df = pd.read_pickle('../../data/df_cont.pkl') 
\end{lstlisting}

\begin{lstlisting}[language=Python]
df_norm = preprocess.MaxScale_df(df).reset_index(drop=True)  # each spectrum is scaled to 1
# reduce size to 1024 and add relative noise
df_noise = df_norm[::2].apply(lambda x: x+x*np.random.normal(0, np.random.randint(1,3)*0.01 , len(x)))
df_scaled = df_noise.T
df_scaled = df_scaled.dropna()
df_scaled = df_scaled.T.reset_index(drop=True)
\end{lstlisting}

\begin{lstlisting}[language=Python]
df_scaled.to_pickle('../../data/df_cont_scaled.pkl')  # save the normalized, scaled df
\end{lstlisting}

\hypertarget{top-layer-data-1}{%
\subsubsection*{Top Layer data}\label{top-layer-data-1}}

\begin{lstlisting}[language=Python]
df_scaled = pd.read_pickle('../../data/df_cont_scaled.pkl')
\end{lstlisting}

\begin{lstlisting}[language=Python]
x_train, x_test, y_train, y_test = train_test_split(df_scaled.T.values,
                                                    df_scaled.columns.map(lambda x: x.split('_')[0]), # first part of the filename is the top label
                                                    test_size=test_size_ratio,
                                                    random_state=42)
\end{lstlisting}

\begin{lstlisting}[language=Python]
y_train = np.array([    
                        [
                            mlb.transform([[y_train[i]]])[0]
                        ] 
                            for i in range(len(y_train))
                        ])
y_test = np.array([ 
                       [
                            mlb.transform([[y_test[i]]])[0],
                       ] 
                            for i in range(len(y_test))
                        ])
\end{lstlisting}

\begin{lstlisting}[language=Python]
data = {
        'name': 'contaminated two-layer and one-layer systems, top labels',
        'x_train': x_train,
        'x_test': x_test,
        'y_train': y_train,
        'y_test': y_test
}
\end{lstlisting}

\begin{lstlisting}[language=Python]
pickle.dump(data, open('../../data/training_data/1/dataset_cont_top_layer.pkl', 'wb'))
\end{lstlisting}

\hypertarget{bot-layer-data-1}{%
\subsubsection*{Bot Layer data}\label{bot-layer-data-1}}

\begin{lstlisting}[language=Python]
df_scaled = pd.read_pickle('../../data/df_cont_scaled.pkl')
\end{lstlisting}

\begin{lstlisting}[language=Python]
x_train, x_test, y_train, y_test = train_test_split(df_scaled.T.values,
                                                    df_scaled.columns.map(lambda x: x.split('_')[1]), # second part of the filename is the bot label
                                                    test_size=test_size_ratio,
                                                    random_state=42)
\end{lstlisting}

\begin{lstlisting}[language=Python]
y_train = np.array([    
                        [
                            mlb.transform([[y_train[i]]])[0]
                        ] 
                            for i in range(len(y_train))
                        ])
y_test = np.array([ 
                       [
                            mlb.transform([[y_test[i]]])[0],
                       ] 
                            for i in range(len(y_test))
                        ])
\end{lstlisting}

\begin{lstlisting}[language=Python]
data = {
        'name': 'contaminated two-layer and one-layer systems, bot labels',
        'x_train': x_train,
        'x_test': x_test,
        'y_train': y_train,
        'y_test': y_test
}
\end{lstlisting}

\begin{lstlisting}[language=Python]
pickle.dump(data, open('../../data/training_data/1/dataset_cont_bot_layer.pkl', 'wb'))
\end{lstlisting}

\hypertarget{clean}{%
\subsection*{clean}\label{clean}}

\begin{lstlisting}[language=Python]
folders = [
            'grad_NO',              # depth-profiles without CO-adv.    with gradient layers
            'sep_NO',               # depth-profiles without CO-adv.    with separated layers
            'one_layer',            # one-layer simulations
            ]

files = [file for file in glob.glob(f'../../data/simulation_data/{folders[0]}/*.spc')]
for folder in folders[1:]:
    files.extend([file for file in glob.glob(f'../../data/simulation_data/{folder}/*.spc')])
\end{lstlisting}

\begin{lstlisting}[language=Python]
df = pd.concat([pd.read_csv(file,
                            sep='\s+', header=None, skiprows=1,
                            usecols=[1],
                            names=['_'.join(file.split('\\')[1].split('_')[:-1])]).T 
                                    for file in files]).T
df.to_pickle('../../data/df_clean.pkl') # save the df without preprocessing
\end{lstlisting}

\hypertarget{preprocess-2}{%
\subsubsection*{preprocess}\label{preprocess-2}}

\begin{lstlisting}[language=Python]
df = pd.read_pickle('../../data/df_clean.pkl') 
\end{lstlisting}

\begin{lstlisting}[language=Python]
df_norm = preprocess.MaxScale_df(df).reset_index(drop=True)  # each spectrum is scaled to 1
# reduce size to 1024 and add relative noise
df_noise = df_norm[::2].apply(lambda x: x+x*np.random.normal(0, np.random.randint(1,3)*0.01 , len(x)))
df_scaled = df_noise.T
df_scaled = df_scaled.dropna()
df_scaled = df_scaled.T.reset_index(drop=True)
\end{lstlisting}

\begin{lstlisting}[language=Python]
df_scaled.to_pickle('../../data/df_clean_scaled.pkl')  # save the normalized, scaled df
\end{lstlisting}

\hypertarget{top-layer-data-2}{%
\subsubsection*{Top Layer data}\label{top-layer-data-2}}

\begin{lstlisting}[language=Python]
df_scaled = pd.read_pickle('../../data/df_clean_scaled.pkl')
\end{lstlisting}

\begin{lstlisting}[language=Python]
x_train, x_test, y_train, y_test = train_test_split(df_scaled.T.values,
                                                    df_scaled.columns.map(lambda x: x.split('_')[0]), # first part of the filename is the top label
                                                    test_size=test_size_ratio,
                                                    random_state=42)
\end{lstlisting}

\begin{lstlisting}[language=Python]
y_train = np.array([    
                        [
                            mlb.transform([[y_train[i]]])[0]
                        ] 
                            for i in range(len(y_train))
                        ])
y_test = np.array([ 
                       [
                            mlb.transform([[y_test[i]]])[0],
                       ] 
                            for i in range(len(y_test))
                        ])
\end{lstlisting}

\begin{lstlisting}[language=Python]
data = {
        'name': 'clean two-layer and one-layer systems, top labels',
        'x_train': x_train,
        'x_test': x_test,
        'y_train': y_train,
        'y_test': y_test
}
\end{lstlisting}

\begin{lstlisting}[language=Python]
pickle.dump(data, open('../../data/training_data/1/dataset_clean_top_layer.pkl', 'wb'))
\end{lstlisting}

\hypertarget{bot-layer-data-2}{%
\subsubsection*{Bot Layer data}\label{bot-layer-data-2}}

\begin{lstlisting}[language=Python]
df_scaled = pd.read_pickle('../../data/df_clean_scaled.pkl')
\end{lstlisting}

\begin{lstlisting}[language=Python]
x_train, x_test, y_train, y_test = train_test_split(df_scaled.T.values,
                                                    df_scaled.columns.map(lambda x: x.split('_')[1]), # second part of the filename is the bot label
                                                    test_size=test_size_ratio,
                                                    random_state=42)
\end{lstlisting}

\begin{lstlisting}[language=Python]
y_train = np.array([    
                        [
                            mlb.transform([[y_train[i]]])[0]
                        ] 
                            for i in range(len(y_train))
                        ])
y_test = np.array([ 
                       [
                            mlb.transform([[y_test[i]]])[0],
                       ] 
                            for i in range(len(y_test))
                        ])
\end{lstlisting}

\begin{lstlisting}[language=Python]
data = {
        'name': 'clean two-layer and one-layer systems, bot labels',
        'x_train': x_train,
        'x_test': x_test,
        'y_train': y_train,
        'y_test': y_test
}
\end{lstlisting}

\begin{lstlisting}[language=Python]
pickle.dump(data, open('../../data/training_data/1/dataset_clean_bot_layer.pkl', 'wb'))
\end{lstlisting}

\hypertarget{training-data-with-multi}{%
\section*{Training Data with Multi}\label{training-data-with-multi}}

\hypertarget{training-data}{%
\subsection*{Training Data}\label{training-data}}

\hypertarget{load-data-and-build-dataframe}{%
\subsection*{Load data and build
dataframe}\label{load-data-and-build-dataframe}}

\begin{lstlisting}[language=Python]
folders = [
            'multi_one_layer'
            ]

files = [file for file in glob.glob(f'../../data/simulation_data/{folders[0]}/*.spc')]
\end{lstlisting}

\begin{lstlisting}[language=Python]
# windows
df = pd.concat([pd.read_csv(file,
                            sep='\s+', header=None, skiprows=1,
                            usecols=[1],
                            names=['_'.join(file.split('\\')[1].split('_')[:-1])]).T 
                                    for file in files]).T
df.to_pickle('../../data/df_multi.pkl') # save the df without preprocessing
\end{lstlisting}

\hypertarget{preprocess-3}{%
\subsection*{Preprocess}\label{preprocess-3}}

\begin{lstlisting}[language=Python]
df = pd.read_pickle('../../data/df_multi.pkl')
\end{lstlisting}

\begin{lstlisting}[language=Python]
df_norm = preprocess.MaxScale_df(df).reset_index(drop=True)                                                    # each spectrum is scaled to 1
df_pp_noise = df_norm[::2].apply(lambda x:  x+x*np.random.normal(0, np.random.randint(1,3)*0.01 , len(x)))     # reduce size to 1024 and add noise
df_scaled = df_pp_noise.T
df_scaled= df_scaled.dropna()
df_scaled = df_scaled.T.reset_index(drop=True)
\end{lstlisting}

\begin{lstlisting}[language=Python]
df_scaled.to_pickle('../../data/df_multi_scaled.pkl')  # save the normalized, scaled df
\end{lstlisting}

\hypertarget{transform-data}{%
\subsection*{Transform data}\label{transform-data}}

\begin{lstlisting}[language=Python]
df_scaled = pd.read_pickle('../../data/df_scaled.pkl')
\end{lstlisting}

\begin{lstlisting}[language=Python]
x_train, x_test, y_train, y_test = train_test_split(df_scaled.T.values,
                                                    df.columns.map(lambda x: x.split('_')[:-1]
                                                                   ).map(base.pair_list_to_tuples
                                                                         ).map(base.one_hot_encode_concentrations),
                                                    test_size=0.3,
                                                    random_state=42)
\end{lstlisting}

\begin{lstlisting}[language=Python]
y_train = np.array([y_train])
y_test = np.array([y_test])
\end{lstlisting}

\begin{lstlisting}[language=Python]
data = {
        'name': 'mixed systems, one layer',
        'x_train': x_train,
        'x_test': x_test,
        'y_train': y_train,
        'y_test': y_test
}
\end{lstlisting}

\begin{lstlisting}[language=Python]
pickle.dump(data, open('../../data/dataset_multi.pkl', 'wb'))
\end{lstlisting}

\hypertarget{training-data-with-depth}{%
\section*{Training Data with Depth}\label{training-data-with-depth}}

\begin{lstlisting}[language=Python]
df_scaled = pd.read_pickle('../../data/df_scaled.pkl')
\end{lstlisting}

\begin{lstlisting}[language=Python]y

\begin{lstlisting}[language=Python]
# 100, 90, 80, 70, 60 Etching
# 100, 90, 80, 70, 60, Angstrom

# Map the concentration to the corresponding label between 0 and 1
\end{lstlisting}

\begin{lstlisting}[language=Python]
layer_number = 5 # for the depth-profiling simplified in 5 categories: 0-10, 10-20, 20-30, 30-40, 40-50 Angstrom
gradient_bool: bool = False # is a measurement a gradient or not?
layers = [10, 20, 30, 40, 50]

def transform_depth_label(label):
    import math
    if len(label.split('_')) < 4:
        top, bot, depth = label.split('_')
        gradient = 0
    else:
        top, bot, depth, gradient = label.split('_')
    # gives top, bottom, etching-depth, where top starts at 100% (with 0 etching) on the top
    if gradient == 'separate':
        # gradient = False
        gradient = 0
        layer_thickness = (layers.index(int(depth)))
    else:
        # gradient = True
        gradient = 1
        depth = int(depth)
        if depth <= 60:
            layer_thickness = 0.0
        elif depth > 100:
            layer_thickness = 1.0
        else:
            layer_thickness = (depth - 60) / 40.0

    return np.array([gradient, layer_thickness], dtype=np.float32)
\end{lstlisting}

\begin{lstlisting}[language=Python]
x_train, x_test, y_train, y_test = train_test_split(df_scaled.T.values,
                                                    df_scaled.columns.map(lambda x: transform_depth_label(x)), # first part of the filename is the top label
                                                    test_size=0.3,
                                                    random_state=42)
\end{lstlisting}

\begin{lstlisting}[language=Python]
data = {
        'name': 'depth profile labels',
        'x_train': x_train,
        'x_test': x_test,
        'y_train': y_train,
        'y_test': y_test
}
\end{lstlisting}

\begin{lstlisting}[language=Python]
pickle.dump(data, open('../../data/dataset_depth.pkl', 'wb'))
\end{lstlisting}


\subsection{Test dataset creation \& preprocessing}
\label{train_data_generation}
\begin{lstlisting}[language=Python]
import matplotlib.pyplot as plt
import pandas as pd
import sys
sys.path.append('../../modules')
import preprocess
\end{lstlisting}

\hypertarget{experimental-data}{%
\subsubsection{Experimental data}\label{experimental-data}}

\begin{lstlisting}[language=Python]
import os
path = '../../data/test_data/Selected_Spectra/elemental'
elemental_exp = os.listdir(path)
\end{lstlisting}

\begin{lstlisting}[language=Python]
# concatenate all elemental spectra into one dataframe
import numpy as np
exp_df = pd.DataFrame()
for file in elemental_exp:
    filepath = os.path.join(path, file)
    try:
        x, y, x_new, y_new, label = preprocess.parse_file(filepath, filetype='vms')
    except: continue
    if x_new is not None:
        exp_df = pd.concat([exp_df, pd.DataFrame(np.flip(y_new), columns=['_'.join(label)])], axis=1)
\end{lstlisting}

\begin{lstlisting}[language=Python]
exp_df.shape
\end{lstlisting}

\begin{lstlisting}
(1024, 214)
\end{lstlisting}

\begin{lstlisting}[language=Python]
exp_df # 213 spectra with elemental composition
exp_df.to_pickle('../../data/experimental_data_elemental.pkl')
\end{lstlisting}


\section{Models}
\subsection{Task 1}

\subsection{Task 2}

\subsection{Task 3}
\section{Training Notebook}

\section{Prediction Notebook}


% Appendix: Declaration of Originality

\chapter{Declaration of Originality} % Main appendix title

\label{AppendixA} % For referencing this appendix elsewhere, use \ref{AppendixA}

\includepdf[pages=-]{Appendices/plagiatserklaerung-master-eng.pdf}
 % from https://www.zhaw.ch/en/lsfm/study/studiweb/master-ls/masters-thesis/
%% !TEX root = ../main.tex

%----------------------------------------------------------------------------------------
% APPENDIX TEMPLATE
%----------------------------------------------------------------------------------------

\chapter{Confusion matrices} % Main appendix title

\label{AppendixB} % Change X to a consecutive letter; for referencing this appendix elsewhere, use \ref{AppendixX}


\begin{center}
\begin{figure}
        \includegraphics[width=\textwidth]{Figures/CM_CNN_1L.png}
    \centering
    \caption{Confusion Matrix of Test-Data for CNN Top-Layer prediction}
    \label{cm_cnn_1l}
\end{figure}
\end{center}


%\include{Appendices/AppendixC}

%----------------------------------------------------------------------------------------
%	BIBLIOGRAPHY
%----------------------------------------------------------------------------------------

\printbibliography[heading=bibintoc]

%----------------------------------------------------------------------------------------

\end{document}  
